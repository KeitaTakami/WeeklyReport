%\documentstyle[epsf,twocolumn]{jarticle}       %LaTeX2.09仕様
\documentclass[twocolumn,a4paper,dvipdfmx]{ujarticle}     %pLaTeX2e仕様
%%%%%%%%%%%%%%%%%%%%%%%%%%%%%%%%%%%%%%%%%%%%%%%%%%%%%%%%%%%%%%
%%
%%  基本 バージョン
%%
%%%%%%%%%%%%%%%%%%%%%%%%%%%%%%%%%%%%%%%%%%%%%%%%%%%%%%%%%%%%%%%%
\setlength{\topmargin}{-45pt}
%\setlength{\oddsidemargin}{0cm} 
\setlength{\oddsidemargin}{-7.5mm}
%\setlength{\evensidemargin}{0cm} 
\setlength{\textheight}{24.1cm}
%setlength{\textheight}{25cm} 
\setlength{\textwidth}{17.4cm}
%\setlength{\textwidth}{172mm} 
\setlength{\columnsep}{11mm}

\kanjiskip=.07zw plus.5pt minus.5pt


%【節がかわるごとに(1.1)(1.2) …(2.1)(2.2)と数式番号をつけるとき】
%\makeatletter
%\renewcommand{\theequation}{%
%\thesection.\arabic{equation}} %\@addtoreset{equation}{section}
%\makeatother

%\renewcommand{\arraystretch}{0.95} 行間の設定

%%%%%%%%%%%%%%%%%%%%%%%%%%%%%%%%%%%%%%%%%%%%%%%%%%%%%%%%
\usepackage[dvipdfm]{graphicx}   %pLaTeX2e仕様(要\documentstyle ->\documentclass)
%%%%%%%%%%%%%%%%%%%%%%%%%%%%%%%%%%%%%%%%%%%%%%%%%%%%%%%%
\usepackage{listings}
\usepackage{grffile}
\usepackage{comment}
\usepackage{here} 
\usepackage{csvsimple}
\usepackage{float}

\begin{document}

\twocolumn[
\noindent
\hspace{1em}

2021年1月22日(金)ゼミ資料
\hfill
\ \ M1 高見啓太

\vspace{2mm}
\hrule
\begin{center}
{\Large \bf 進捗報告}
\end{center}
\hrule
\vspace{3mm}
]

\section{今週行ったこと}
\begin{itemize}
	
	\item VGG16のモデルを転移学習させて,猫に耳カットがあるか否かの識別を行った.
	\section{耳カットの実験}
	VGG16を転移学習させて,猫の耳カットを識別させるモデルを作った.表\ref{tab:mimicut}にモデルのパラメータを示す.クラスとしては,耳カットなし,あり,不明の3クラスとなる.画像の中から一番少ないnoncutのデータ枚数に合わせて実験を行った.また,バッチサイズは16とした.図\ref{cutaccuracy},図\ref{cutloss}に各クラス39枚の時のaccuracy,lossを,図\ref{cutaccuracy2},図\ref{cutloss2}に各クラス110枚の時のaccuracy,lossをそれぞれ示す.
\end{itemize}
\begin{table}[htb]
	\begin{center}
		
		\caption{耳カット識別のモデル\label{tab:mimicut}}
		\begin{tabular}{|c|c|} \hline
			クラス & 3クラス分類 \\ \hline \hline
			訓練データ数 & 各クラス39枚/110枚\\
			input & image(224×224×3) \\
			output & class(3) \\
			ベースモデル & VGG16 \\
			optimizer & adam \\
			学習率 & 0.001\\
			損失関数 & categorical\_crossentropy\\ 
			train:validation & 2:1\\
			初期重み & ImageNet \\
			batch\_size & 16 \\
			epochs & 30 \\ \hline
		\end{tabular}
	\end{center}
\end{table}


\begin{figure}[H]
	\begin{center}
		\includegraphics[width=7cm]{2021_01_15_cat_earcut_accuracy_batchsize_16_adam_lr0.001.png}
	\end{center}
	\caption{耳カット識別のaccuracyの推移(各クラス39枚)\label{cutaccuracy}}
\end{figure}

\begin{figure}[H]
	\begin{center}
		\includegraphics[width=7cm]{2021_01_15_cat_earcut_loss_batchsize_16_adam_lr0.001.png}
	\end{center}
	\caption{耳カット識別のlossの推移(各クラス39枚)\label{cutloss}}
\end{figure}

\begin{figure}[H]
	\begin{center}
		\includegraphics[width=7cm]{2021_01_22_cat_earcut_accuracy_batchsize_16_adam_lr0.001.png}
	\end{center}
	\caption{耳カット識別のaccuracyの推移(各クラス110枚)\label{cutaccuracy2}}
\end{figure}

\begin{figure}[H]
	\begin{center}
		\includegraphics[width=7cm]{2021_01_22_cat_earcut_loss_batchsize_16_adam_lr0.001.png}
	\end{center}
	\caption{耳カット識別のlossの推移(各クラス110枚)\label{cutloss2}}
\end{figure}

訓練枚数を増やすと識別率は多少上がった.


\section{次回行うこと}
\begin{itemize}
	\item アノテーションの続き
	\item 猫の耳を検出できるか実験
\end{itemize}





%\bibliography{index}
\bibliographystyle{junsrt} %参考文献出力スタイル

\end{document}

